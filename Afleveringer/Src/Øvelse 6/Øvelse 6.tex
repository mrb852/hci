\documentclass[12pt,a4paper,oneside]{article}
\usepackage{amssymb}
\usepackage{amsthm}
\usepackage[utf8]{inputenc}
\usepackage[danish]{babel}
\usepackage{mathtools}
\usepackage{fancyhdr}
\usepackage[danish]{isodate}
\usepackage{graphicx}
\usepackage{color}

\definecolor{gray}{rgb}{0.5,0.5,0.5}
\usepackage{listings}

\setlength{\parindent}{0pt}

\author{Bertram A. Nicolas, Christian Enevoldsen}

\newcommand{\Opgn}{6}


\lstset{
  frame=tb,
  language=ML,
  aboveskip=3mm,
  belowskip=3mm,
  showstringspaces=false,
  columns=flexible,
  basicstyle={\small\ttfamily},
  numbers=left,
  numberstyle=\tiny,
  keywordstyle=\color{blue},
  commentstyle=\color{gray},
  stringstyle=\color{red},
  breaklines=true,
  breakatwhitespace=true
  tabsize=4
}

\pagestyle{fancy}
\fancyhf{}
\setlength{\headheight}{2.5em}
\setlength{\headsep}{3em}
\fancyhead[R]{\small \Opgn. hjemmeopgave; 2014 HCI \\
Bertram André Nicolas \& Christian Enevoldsen}
\renewcommand{\headrulewidth}{0pt}

\fancyfoot[L]{\today}
\fancyfoot[C]{\thepage}
\fancyfoot[R]{}

\renewcommand{\thesubsection}{}
\renewcommand{\thesubsubsection}{$\bullet${}}
\begin{document}

\section*{Drejebog:}

\subsection{1. Startbetingelser}
\begin{itemize}
\item{Brugeren skal sidde foran en computer der har et optage program der optager lyd fra en mikrofon og skærmen.
}
\item{Computeren skal sættes til at optage og så derefter vise siden "about:blank" i en "inkognito"-browser vindue.
}

\end{itemize}

\subsection{2. Tjekliste til briefing}

\begin{itemize}
\item{Jeg er neutral - jeg har ikke deltaget i udviklingen af webstedet.}

\item{Du skal ikke holde dig tilbage med kritik}
\item{Vi tester ikke dig men webstedet}
\item{Jeg vil bede dig at tænke højt under testen, så jeg kan følge med i de overvejelser, du gør dig.}
\item{Jeg vil gerne have dine ufiltrerede kommentarer, både ris og ros. Alle kommentarer er interessante.}
\item{Har du nogen spørgsmål?}
\end{itemize}

\subsection{3. Interview før testsession}

\textbf{Personlige oplysninger}

- Navn

- Alder

- Profession


\subsection{4. Testopgaver}

\begin{enumerate}

\item{ Find hjemmesiden ud fra basale informationer}

\textbf{Opgave:} Din ven har fået en faldskærmsudsprings oplevelse til sin fødselsdag, du ønsker selv samme oplevelse

\textbf{Løst:} når testdeltageren har fundet oplevelsesgaver.dk

\item{Find en oplevelse der passer til dig}

\textbf{Opgave:} Find en oplevelse du kunne tænke dig at købe, den må maksimalt koste 500 kr.

\textbf{Løst:} Når testdeltageren har fundet en oplevelse de synes om.

\item{Find en oplevelse der passer til en ven}

\textbf{Opgave:} Din bror/søster har fødselsdag, find en gave som passer til ham/hende

\textbf{Løst:} Når testdeltageren har fundet en gave de mener passer deres bror/søster.

\item{
Find en oplevelse der passer til grupper på mere end 4 personer.}

\textbf{Opgave:} Du har aftalt med dine venner at i skal ud og have en oplevelse. I er i hvert fald 4 der gerne vil med med sikkerhed. Find en oplevelse hvor i ale kan være med, og der ville være plads til flere.

\textbf{Løst:} Når testdeltageren har fundet en oplevelse hvor man kan tage til som en samlet gruppe

\item{Test af køb af oplevelses gave.}

\textbf{Opgave:} Køb et gavekort som du godt kunne tænke dig, du har 500 kr, dette skal være gavekort på mail.

\textbf{Løst:} Når testdeltageren har købt gavekortet.

\item{Ombytning af oplevelsesgave}

\textbf{Opgave:} Byt den oplevelse du lige har købt til den samme oplevelse

\textbf{Løst:} Når testdeltageren har byttet oplevelsen.

\item{Returnering af gavekortet}

\textbf{Opgave:} Det viser sig at du desværre skal bruge pengene et andet sted, returner gavekortet.

\textbf{Løst:} Når testdeltageren har returneret gavekortet - http://www.oplevelsesgaver.dk/kundehenvendelser/fortrydelsesret

\item{Finde relevante oplysninger}

\textbf{Opgave:} Din mor har fået oplevelsesgave til at ?, hun vil gerne vide om hun kan ?

\textbf{Løst:} Når testdeltageren har fundet oplysningen

\item{Finde relevant opgave}

\textbf{Opgave:} Du skal give en oplevelse til en mand, som gerne vil have en smagsoplevelse i København

\textbf{Løst:} Når kunden har fundet en oplevelse

\item{Finde vigtige informationer omkring oplevelse/gavokort.}

\textbf{Opgave:} Du har valgt at købe et vilkårligt gavekort. Hvor lang tid kan det indløses?

\textbf{Løst:} når kunden har fundet ud af at man har 3 år.

\item{Fortrydelsesret}

\textbf{Opgave:} Du er i tvivl om en person vil gøre brug af en oplevelse. Hvad mange dages fortrydelsesret har man?

\textbf{Løst:} Når kunden siger 14 dage.

\item{Tryghed}

\textbf{Opgave:} Du er skeptisk omkring hjemmesiden, og vil sikre dig at du kan handle trygt. Hvordan finder du ud af det?

\textbf{Løst:} Når kunden har tjekket trustpilot eller e-mærket.

\end{enumerate}

\subsection{5. Interview efter test}

\begin{itemize}

\item{Har du før brugt en form for oplevelses gaver, enten fået som gave, eller givet til en anden?}
\item{Hvilke 2-3 ting fungerer bedst på webstedet?}
\item{Hvilke 2-3 ting trænger mest til at blive forbedret?}
\item{Kunne du finde på at bruge oplevelsesgaver.dk?}
\item{Har du før brugt en form for oplevelses gaver, enten fået som gave, eller givet til en anden?}
\item{Hvilke 2-3 ting fungerer bedst på webstedet?}
\item{Hvilke 2-3 ting trænger mest til at blive forbedret?}
\item{Kunne du finde på at bruge oplevelsesgaver.dk?}

\end{itemize}

\end{document}