\documentclass[12pt,a4paper,oneside]{article}
\usepackage{amssymb}
\usepackage{amsthm}
\usepackage[utf8]{inputenc}
\usepackage[danish]{babel}
\usepackage{mathtools}
\usepackage{fancyhdr}
\usepackage[danish]{isodate}
\usepackage{graphicx}
\usepackage{color}
\usepackage{listings}

\setlength{\parindent}{0pt}

\begin{document}

\title{Menneske-datamaskine interaktion\\Eksamen\\2014}
\author{Christian Hohlmann Enevoldsen, MRB852}
\date{\today}
\maketitle
\newpage

% Opgave 1
\section{Brugsvenlighed}

Webstedet lever ikke op til følgende komponenter indenfor brugsvenlighed

\begin{enumerate}
\item{Effektivitet}
\item{Tilgængelighed}

\end{enumerate}

% Opgave 2
\section{Fejlmeddelelse}

\subsection{Forbedringer}
Konkrete forslag til forbedring af fejlmeddelelsen på Skift læge\\

Der bør være en teknisk note til IT folk, så de evt. hurtigt kan identificere fejlen, og skrive fejlkoden til skiftlæge.dk's support. 

"Beklager!": undgå udråbstegnet da det virker som om de råber af brugeren.

Det er godt at der er link til hjælp.

Fejltidspunktet er lidt irrelevant. I stedet bør man have info til hvordan man kommer i kontakt med supporten. Det er sjældent at et fejltidspunkt er rigtig nyttigt. Det er mere relevant ved E-mail afsending/modtagelse fejl.\\

\subsection{Eksempel på en fejlmeddelelse}


\text{Beklager.}\\
\text{Der skete en fejl.}\\
\text{Fejlen opstår som regel når}
\begin{itemize}
\item{Der gik for lang tid mellem interaktioner på hjemmesiden. \underline{prøv igen}} 
\item{Sikkerhedsindstillingerne i din browser er sat forkert. Se hvordan de kan stilles \underline{her}}
\end{itemize}
Du skal være velkommen til at \underline{prøve igen}. Hjælper ovenstående ikke på problemet skal du være velkommen til at ringe til os på 12345678 eller skrive en mail på \underline{support@skiftlæge.dk}
 
% Opgave 3
\section{Tjekliste til interview - review}

\begin{enumerate}
\item{Det er en god indledning, som har en kort briefing og får relevante informationer ud af interviewdeltageren}
\item{Spørgsmål 9. Er et godt spørgsmål da det åbner op til at deltageren kan svare bredt og neutralt}
\item{I spørgmål 10. og dets underspørgsmål, virker det som om det er interviewerens egne gode idéer han/hun remser op. Stort minus}
\item{Nogle af spørgsmålene er ledene. F.eks spørgsmål 5 og 6. Det antages at personen er glad, og intervieweren hinter til, eller prøver at overtale deltageren til, at der mangler info på rejser.}
\item{De indledende kommentarer er gode for at holde overblik over hvad spørgsmålene omhandler.}
\item{Der bliver stillet mange lukkede spørgsmål, hvilket ikke er godt ved interviews, da det leder til korte og uinspirerede svar. Det kan også komme til at lyde som en afhøring i nogle tilfælde.}
\item{Han/hun introducerer ikke sig selv. Det er vigtigt at skabe god rapport, hvis man skal forvente gode og ærlige svar.}
\item{Spørgsmål 13, er ikke et spørgsmål man skal stille til en evt. bruger. Det er noget teknisk en ingienør skal tage bekymre sig om.}
\item{Intervieweren introducerer system før deltageren er kommet med sine egne idéer. Nu kan deltageren blot sige, at det lyder godt og de glæder sig til at det kommer.}

\end{enumerate}

% Opgave 4
\section{Scenarie og persona}

\subsection{Scenarie}

En mand venter på bussen. Han har 12 minutter før bussen kommer. Der er en kiosk ved siden af. Han får en god idé. Han bruger de 12 minutter på at købe noget i kiosken for at få tiden til at gå. Han ærger sig over impulskøbet, men bliver glad når bussen triller forbi.

\subsection{Persona}

Den persona som optræder i billederne er en typisk midt 20, start 30'erne som er rastløs og gerne vil betale prisen for at blive underholdt. Han er en naiv impulskøber, med vaner for at spendere et par kroner her og der. Han er en social person, som ikke er vant til at kede sig eller vente.

\section{Usabilitytest}

\subsection{Uddybende spørgsmål}

\begin{itemize}
\item{Hvor finder man informationerne henne?}
\item{Skal forbindelsen kun være via tog?}
\item{Hvordan ændres prisen i forhold til forbindelsen mellem byer?}
\item{Hvorfor skulle systemet ændre sig fra dato til dato og fra tid til tid?}
\item{Gælder byerne kun i Danmark og evt kun dele af Danmark?}
\end{itemize}
 
\subsection{Usabilitytestopgaver}

\begin{itemize}
\item{Find en forbindelse fra din by til din naboby.}
\item{En person skal fra Brøndby til Høje Taastrup. Hvilen forbindelse er billigst?}
\item{En person ønsker at rejse fra København til Roskilde kl 12:00. Kan det lade sig gøre i morgen? }
\item{En person skal være i Aarhus senest kl 13 imorgen. Personen rejser fra København. Find den billigste rejse, så personen er der senest kl 13 imorgen. Løst når personen har fundet den billigste rejse i tidsrummet 00:01 - 13:00 imorgen.}
\end{itemize}

\section{Positive Resultater}

\begin{itemize}
\item{Det der fungerer godt skal ikke laves om, men derimod værdsættes og vedligeholdes}
\item{En udelukkende negativ rapport kan virke stødende og nedslående.}
\end{itemize}


\newpage

\section{E-mail fra skift læge}

\subsection{Bekræftelsen}

Kære Anne Pedersen\\

Denne mail er en bekræftelse på dit lægeskift på skiftlæge.dk, som du har foretaget dig 01/11-14 kl 10:34.\\

Lægeskiftet træder i kraft d. 01/12-14\\

Informationer om din nye læge\\

Navn: Peder Petersen. (Mand)\\
Adresse: samplevej 20\\
Tlf: 13192033\\
Hjemmeside: pederpetersen.dk\\

Du behøver ikke at foretage der noget, men hvis du har spørgsmål skal du være velkommen til at kontakte os på mail \underline{her} eller ringe til os på 32781021.\\
Information omkring vores åbnings- og svartider findes \underline{her}\\

Tak fordi du valgte at bruge skiftlæge.dk\\

Med venlig hilsen.\\
En supporter

\section{Spørgeskema}

\begin{enumerate}
\item{Det var let at finde trafikinformation}
\item{Jeg mærkede ikke ventetiden}
\item{Jeg følte at skærmens størrelse var tilstrækkelig til alle menneskerne}
\item{Jeg vil bruge skærmen igen}
\end{enumerate}

Det sidste spørgsmål er nok det mest væsentlige i en beta. Hvis folk ikke vil bruge det skal man nok finde på en anden idé, eller finde ud af hvorfor ved en ny mere dybdegående test.

\section{Udviklingsprocess}

Det kan være en fordel at have testdeltagere som ikke er eksperter i sådanne programmer. Der skal dog laves nogle tests som ikke er alt for musik komplicerede. \\

Før der implementeres (delvist) kan man lave papirprototyper, som kan ikke koster så meget tid. Tid er vigtigt når man skriver speciale\\

På nuværende tidspunkt kan jeg godt forstå der ikke vil fokuseres på Windows, men iOS er et must at tjekke. Tænk hvis der ikke var netop din idé, men stor efterspørgsel. Om ikke andet kan man tjekke om nogle af disse musik apps har noget der mangler på android\\

Hvis et delmål er at tjene nogle penge. (I hvert fald bare til omkostningerne) er det værd at spørge dine testdeltagere, hvad de synes vil være den bedste "monetizing" plan. Skal det flyde med reklamer, skal den koste penge eller skal man kunne købe flere redskaber i den gratis version? \\

Lav et interview før der overhovedet tænkes på at implementere noget. Find ud af hvad komponister mangler, og lav det. Der er ingen grund til at lave en kopi af andre apps, hvis ikke de kan noget mere eller er billigere og kan det samme.\\

Lav en uddybende persona af den typiske bruger af appen.
 

\end{document}